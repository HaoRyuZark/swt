\documentclass[12pt]{article}

% Sprache, Encoding & Fonts
\usepackage[utf8]{inputenc}
\usepackage[T1]{fontenc}

% Layout & Grundpakete
\usepackage{geometry}
\geometry{margin=1in}
\usepackage{fullpage} % optional; geometry setzt Margin
\usepackage{setspace}
\usepackage{parskip}

% Mathematik & Symbole
\usepackage{amsmath,amssymb,amsthm}
\usepackage{mathrsfs}

% Grafiken & Plots
\usepackage{graphicx}
\graphicspath{{./resources/images/}}
\usepackage{caption}
\usepackage{subcaption}
\usepackage{float}
\usepackage{pgfplots}
\pgfplotsset{compat=newest}
\usepackage{tikz}
\usetikzlibrary{calc,intersections,decorations.pathreplacing,patterns,angles,quotes,shapes,arrows,positioning,automata}
\usepgfplotslibrary{fillbetween}

% Sonstige
\usepackage{enumitem}
\usepackage{booktabs}
\usepackage{multicol}
\usepackage{multirow}
\usepackage{polynom}
\usepackage{tikz-cd}
\usepackage{comment}
\usepackage{fancyvrb}
\usepackage{fancyhdr}
\usepackage{etoolbox}
\usepackage{listings}
\usepackage{tkz-euclide}
\usepackage{float}
\usepackage{hyperref}
\usepackage{siunitx} % für circuitikz/si-Einheiten falls benötigt
\usepackage{circuitikz} % optional, behält Option ohne Fehler
\usepackage{bookmark}
\usepackage{tabularx}
\usepackage{booktabs}

% Header/Footer
\pagestyle{fancy}
\fancyhead[l]{Gruppe 12}
\fancyhead[c]{SWT \#3}
\fancyhead[r]{\today}
\fancyfoot[c]{\thepage}
\renewcommand{\headrulewidth}{0.2pt}
\setlength{\headheight}{15pt}

% pgfplots etc.
\pgfplotsset{compat=1.18}

\begin{document}

\section*{A2 Scrum}

\subsection*{[a] Vorgehen in zwei Situationen während des Sprints}

\paragraph{Fall 1: Sprintziel und alle User Stories sind \emph{vorzeitig} erfüllbar.}
\begin{enumerate}
\item \textbf{Sprint nicht verlängern oder verkürzen} (Timeboxing beibehalten): Die Sprintdauer ist fix und wird nicht angepasst; nur Ereignisse dürfen enden, sobald ihr Zweck erfüllt ist. 
\item \textbf{Mit dem Product Owner (PO) \emph{Scope neu vereinbaren}:} Das Dev-Team kann, falls sinnvoll, weitere gut vorbereitete Product-Backlog-Einträge (DoR erfüllt) in den Sprint \emph{ziehen}, sofern das Sprintziel nicht gefährdet wird. Inhalt und Umfang des Sprints dürfen mit dem PO neu vereinbart werden. 
\item \textbf{Weitere Inkremente liefern:} Mehrere Inkremente pro Sprint sind möglich; vorzeitige Lieferung ist erlaubt, solange die \emph{Definition of Done} eingehalten wird. 
\item \textbf{Qualität stärken & Schulden abbauen:} Falls keine zusätzlichen, wertvollen Stories verfügbar sind: Refactoring, Testausbau und Dokumentation gemäß DoD priorisieren. 
\item \textbf{Sprintabbruch nur in Ausnahmefällen:} Wird das Sprintziel obsolet, kann ausschließlich der PO den Sprint abbrechen. 
\end{enumerate}

\paragraph{Fall 2: Sprintziel und User Stories sind \emph{sehr wahrscheinlich} nicht rechtzeitig erfüllbar.}
\begin{enumerate}
\item \textbf{Transparenz & Re-Planung im Daily Scrum:} Fortschritt und Hindernisse offenlegen; Plan anpassen, um das Sprintziel bestmöglich zu erreichen. 
\item \textbf{Scope mit dem PO neu verhandeln:} Umfang des Sprint Backlogs \emph{reduzieren} (z.,B.\ niedere Prioritäten verschieben), ohne das Sprintziel oder Qualitätsansprüche zu gefährden. \emph{Qualität nimmt nicht ab}. 
\item \textbf{Impediments beseitigen lassen:} Der Scrum Master unterstützt aktiv beim Entfernen von Hindernissen. 
\item \textbf{DoD nicht aufweichen:} Keine Abstriche bei Qualitätskriterien; unvollständige Arbeit bleibt \enquote{nicht fertig} und wird transparent gemacht. 
\item \textbf{Keine Sprintverlängerung und kein \enquote{Personen nachschieben}:} Die Sprintlänge bleibt fix; zusätzliches Personal macht späte Projekte oft noch später (Brooks’sches Gesetz). 
\item \textbf{Stakeholder-Einbindung:} Ergebnisse und Anpassungsbedarf spätestens im Sprint Review demonstrieren und besprechen. 
\end{enumerate}

\subsection*{[b] Sehr kurze (z.,B.\ 1 Woche) vs.\ sehr lange (z.,B.\ 8 Wochen) Sprintlängen}

\begin{table}[h]
\centering
\renewcommand{\arraystretch}{1.15}
\begin{tabularx}{\textwidth}{@{} l X X @{}}
\toprule
& \textbf{Vorteile} & \textbf{Nachteile} \\
\midrule
\textbf{Kurz (\(\approx\) 1 Woche)} &
\begin{itemize}[leftmargin=*]
  \item Sehr häufiges Feedback; schnelle Kurskorrekturen.
  \item Frühe Risikosichtbarkeit und kurze Lernschleifen.
  \item Hohe Fokusdisziplin auf kleine, klar geschnittene Inkremente.
  \item Regelmäßige Lieferbarkeit und verlässlicher Takt.
\end{itemize}
&
\begin{itemize}[leftmargin=*]
  \item Relativ hoher Event-Overhead (Planning/Review/Retrospektive) im Verhältnis zur Umsetzung.
  \item Gefahr von zu kleinteiligen Stories bzw.\ Stückelung.
  \item Mehr Planungsdruck; häufigere Schätz-/Refinement-Zyklen.
\end{itemize}
\\
\midrule
\textbf{Lang (\(\approx\) 8 Wochen)} &
\begin{itemize}[leftmargin=*]
  \item Weniger „Ceremony“-Overhead pro Arbeitswoche.
  \item Mehr Zeit für komplexe Features innerhalb eines Sprints.
  \item Weniger Kontextwechsel durch selteneres Planen/Reviewen.
\end{itemize}
&
\begin{itemize}[leftmargin=*]
  \item Weniger Feedback; Risiken werden später sichtbar.
  \item Größere Batches/WIP $\rightarrow$ spätere Integration, höhere Fehlerrisiken.
  \item Geringere Agilität; Scope Creep wird später erkannt.
\end{itemize}
\\
\bottomrule
\end{tabularx}
\caption{Vier-Felder-Tafel: kurze vs.\ lange Sprintlängen.}
\end{table}

\section*{Aufgabe 4 - UML-Klassendiagramm/Zustandsdiagramm: Stapelspeicher}

\end{document}
