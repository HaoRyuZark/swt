\documentclass[12pt]{article}

% -----------------------------
% Sprache, Encoding & Fonts
% -----------------------------
\usepackage[utf8]{inputenc}
\usepackage[T1]{fontenc}
\usepackage[ngerman]{babel}
\usepackage{csquotes}     % \enquote
\usepackage{microtype}    % bessere Laufweite/Umbruch
\usepackage{ebgaramond}

% -----------------------------
% Layout & Grundpakete
% -----------------------------
\usepackage{geometry}
\geometry{margin=1in}
\usepackage{setspace}
\usepackage{parskip}

% -----------------------------
% Mathematik & Symbole
% -----------------------------
\usepackage{amsmath,amssymb,amsthm}
\usepackage{mathrsfs}

% -----------------------------
% Grafiken & Plots
% -----------------------------
\usepackage{graphicx}
\graphicspath{{./resources/images/}}
\usepackage{caption}
\usepackage{subcaption}
\usepackage{float}
\usepackage{pgfplots}
\pgfplotsset{compat=1.18}
\usepackage{tikz}
\usetikzlibrary{calc,intersections,decorations.pathreplacing,patterns,angles,quotes,shapes,arrows,positioning,automata}
\usepgfplotslibrary{fillbetween}
\usepackage[output=png]{plantuml} % <— PlantUML für UML-Diagramme

% -----------------------------
% Sonstige Nützliches
% -----------------------------
\usepackage{enumitem}
\usepackage{booktabs}
\usepackage{multicol}
\usepackage{multirow}
\usepackage{polynom}
\usepackage{tikz-cd}
\usepackage{comment}
\usepackage{fancyvrb}
\usepackage{fancyhdr}
\usepackage{etoolbox}
\usepackage{listings}
\usepackage{tkz-euclide}
\usepackage{hyperref}
\usepackage{siunitx}   % für circuitikz/Si-Einheiten
\usepackage{circuitikz}
\usepackage{bookmark}
\usepackage{tabularx}
\usepackage{ragged2e}  % für RaggedRight in Tabellenspalten

% -----------------------------
% Tabellen: neue Spaltenart (ragged) für bessere Umbrüche
% -----------------------------
\newcolumntype{Y}{>{\RaggedRight\arraybackslash}X}

% -----------------------------
% Header/Footer
% -----------------------------
\pagestyle{fancy}
\fancyhead[l]{Gruppe 12}
\fancyhead[c]{SWT \#5}
\fancyhead[r]{\today}
\fancyfoot[c]{\thepage}
\renewcommand{\headrulewidth}{0.2pt}
\setlength{\headheight}{15pt}

% Etwas mehr Flexibilität beim Zeilenumbruch
\emergencystretch=1.5em

\begin{document}

\section*{A1 Git}

\textbf{Beginner:}

\begin{itemize}
  
  \item git init 

  \item git status 

  \item git clone it clone https://github.com/user/repo.git, cd repo

  \item git init, git add 

  \item git commit -m Ahhhhhhhhhhh

  \item git rm temp.txt 

  \item git branch -a

  \item git branch feature, git switch -c feature 
  
  \item git switch feature 

  \item git checkout feature 

  \item git checkout -b feature

  \item git remote add mom aaa

  \item git add ., git commit -m aaa, git push -u origin main

  \item git checkout -b 69, git push -u origin 69 

\end{itemize}

\textbf{Advanced:}

\begin{itemize}
    
  \item git add .

  \item git checkout -b develop,  git switch feature, git merge develop 

  \item git switch develop, git merge main 
  
  \item git merge --abort

  \item git switch -c feature/user-auth, git add ., git commit, git push -u origin main, git switch main, git merge feature/user-auth 

  \item git switch -c hotfix/security, git add., git commit -m aa, git switch main, git merge hotfix/security-patch

  \item git switch -c release/2.0.0, git add ., git commit -m "Prepare release 2.0.0", git switch main, git merge release/2.0.0, git tag v2.0.0

  \item git pull

  \item git add /src/auth/login.js, git commit -m "Improve password validation requirements, git pull origin main, git add ., git commit -m "Merge Sarah's email validation with my password improvements"

  \item git switch -c feature/password-reset,git add ., git commit -m "Add password reset functionality", git push origin feature/password-reset

  \item 3x(git reset --soft) 

  \item  3x(git reset --soft)

  \item git log, git reset 

  \item git switch -c hotfix, git switch -c feature, git stash pop

  \item git stash, git switch main, git switch -c feature, git switch feature/old-task, git stash pop

\end{itemize}

\begin{figure}[h]
  \centering
  \includegraphics[width=0.7\textwidth]{beweis.jpg}
  \includegraphics[width=0.7\textwidth]{beweis2.png}
\end{figure}


\newpage
\section*{A2 Herausforderungen}

\textbf{[a]}  
Datei zum letzten Commit hinzufügen, ohne neuen Commit im Verlauf zu erzeugen:
\begin{verbatim}
git add <datei>
git commit --amend --no-edit
\end{verbatim}

\textbf{[b]}  
Commit aus dem lokalen Verlauf vollständig löschen:
\begin{verbatim}
git reset --hard HEAD~1
\end{verbatim}

\textbf{[c]}  
Letzten Commit im Remote-Repository rückgängig machen, Rücknahme bleibt sichtbar:
\begin{verbatim}
git revert HEAD
git push origin <branch>
\end{verbatim}

\textbf{[d]}  
Unterschiede zwischen main und feature seit dem gemeinsamen Vorgänger anzeigen:
\begin{verbatim}
git diff main...feature
\end{verbatim}

\begin{figure}[h]
  \centering
  \begin{minipage}{0.48\textwidth}
    \centering
    \includegraphics[width=\textwidth]{orly_H6.png}
  \end{minipage}\hfill
  \begin{minipage}{0.48\textwidth}
    \centering
    \includegraphics[width=\textwidth]{git2.png}
  \end{minipage}
\end{figure}

\end{document}
