\documentclass[12pt]{article}

% -----------------------------
% Sprache, Encoding & Fonts
% -----------------------------
\usepackage[utf8]{inputenc}
\usepackage[T1]{fontenc}
\usepackage[ngerman]{babel}
\usepackage{csquotes}
\usepackage{microtype}
\usepackage{ebgaramond}

% -----------------------------
% Layout & Grundpakete
% -----------------------------
\usepackage{geometry}
\geometry{margin=1in}
\usepackage{setspace}
\usepackage{parskip}

% -----------------------------
% Mathematik & Symbole
% -----------------------------
\usepackage{amsmath,amssymb,amsthm}
\usepackage{mathrsfs}

% -----------------------------
% Sonstige Nützliches
% -----------------------------
\usepackage{enumitem}
\usepackage{booktabs}
\usepackage{tabularx}
\usepackage{ragged2e}
\usepackage{fancyhdr}
\usepackage{hyperref}
\usepackage{minted}

% -----------------------------
% Tabellen: neue Spaltenart
% -----------------------------
\newcolumntype{Y}{>{\RaggedRight\arraybackslash}X}

% -----------------------------
% Header/Footer
% -----------------------------
\pagestyle{fancy}
\fancyhead[l]{Gruppe 12}
\fancyhead[c]{SWT \#8}
\fancyhead[r]{\today}
\fancyfoot[c]{\thepage}
\renewcommand{\headrulewidth}{0.2pt}
\setlength{\headheight}{15pt}

\emergencystretch=1.5em

\begin{document}

\section*{Aufgabe A1 -- Mobile Application}

\subsection*{a) Klassenverband (Factory Method)}
\begin{itemize}[leftmargin=*]
    \item \textbf{mobileapplication.ui.MobileElement} (abstract, Produkt): schreibgeschütztes Attribut \texttt{title:String}, Referenz auf \texttt{MobileForm}, Operationen \texttt{register(form)}, \texttt{render()} (abstrakt), Getter \texttt{getTitle()}.
    \item \textbf{mobileapplication.ui.MobileForm} (abstract, Creator): Aggregiert \texttt{List<MobileElement> elements}; Factory Method \texttt{createElement(title)} (\textit{protected abstract}), konkrete Methode \texttt{addElement(title)} ruft \texttt{createElement}, registriert Element und legt es in \texttt{elements} ab; \texttt{render()} delegiert an alle Elemente.
    \item \textbf{mobileapplication.ui.AndroidForm / IOSForm} (konkrete Creator): überschreiben \texttt{createElement}, erzeugen jeweils passendes konkretes Element.
    \item \textbf{mobileapplication.ui.AndroidElement / IOSElement} (konkrete Product): besitzen Paket-Sicht-Konstruktor, damit Erzeugung nur über \texttt{MobileForm} möglich ist; implementieren \texttt{render()}.
    \item \textbf{mobileapplication.App}: Attribut \texttt{form: MobileForm}; \texttt{getOS()} liefert \enquote{Android} oder \enquote{iOS}; \texttt{initialize()} legt passende konkrete Form an; \texttt{execute()} fügt drei Elemente hinzu und rendert.
\end{itemize}
Beziehungen: \texttt{MobileForm} ist abstrakter Creator und kennt die Factory Method \texttt{createElement}; konkrete Formen liefern passende konkrete Elemente. \texttt{addElement} kapselt die Erzeugung (Konstruktoren der konkreten Elemente sind paket-privat), sodass von außen keine widersprüchlichen Kombinationen angelegt werden können. \texttt{App} nutzt nur das abstrakte Interface und bleibt vom konkreten OS entkoppelt.

\subsection*{b) Sequenzdiagramm (Android)}
Gefundene Nachricht: \texttt{App.execute()}.
\begin{enumerate}[leftmargin=*,label=\arabic*.]
    \item \texttt{execute()} ruft \texttt{initialize()}.
    \item \texttt{initialize()} erzeugt \texttt{AndroidForm} und speichert sie in \texttt{form}.
    \item Dreimal \texttt{form.addElement("...")}: Aufruf von \texttt{AndroidForm.createElement()}, Konstruktion \texttt{new AndroidElement(title)}, \texttt{register(form)}, Ablage in \texttt{elements}.
    \item \texttt{form.render()}: iteriert über alle \texttt{AndroidElement}-Instanzen, die jeweils \texttt{render()} ausführen.
\end{enumerate}
Die Nachrichten laufen zwischen den konkreten Klassen \texttt{App} $\rightarrow$ \texttt{AndroidForm} $\rightarrow$ \texttt{AndroidElement}; dank Factory Method entsteht keine Kreuzung Android/iOS.

\subsection*{c) Java-Rumpf (Paketstruktur angedeutet)}
\begin{minted}[fontsize=\footnotesize,breaklines,linenos]{java}
// mobileapplication/ui/MobileElement.java
package mobileapplication.ui;

public abstract class MobileElement {
    private final String title;
    private MobileForm form;

    protected MobileElement(String title) { // nur Paket/SUB-Klassen
        this.title = title;
    }

    public String getTitle() { return title; }

    void register(MobileForm form) { // Paket-sichtbar
        this.form = form;
    }

    public MobileForm getForm() { return form; }

    public abstract void render();
}

// mobileapplication/ui/MobileForm.java
package mobileapplication.ui;

import java.util.ArrayList;
import java.util.List;

public abstract class MobileForm {
    private final List<MobileElement> elements = new ArrayList<>();

    protected abstract MobileElement createElement(String title);

    public final void addElement(String title) {
        MobileElement element = createElement(title);
        element.register(this);
        elements.add(element);
    }

    public void render() {
        elements.forEach(MobileElement::render);
    }
}

// mobileapplication/ui/AndroidElement.java
package mobileapplication.ui;

class AndroidElement extends MobileElement {
    AndroidElement(String title) { super(title); }

    @Override
    public void render() {
        System.out.println("Android-Element: " + getTitle());
    }
}

// mobileapplication/ui/IOSElement.java
package mobileapplication.ui;

class IOSElement extends MobileElement {
    IOSElement(String title) { super(title); }

    @Override
    public void render() {
        System.out.println("iOS-Element: " + getTitle());
    }
}

// mobileapplication/ui/AndroidForm.java
package mobileapplication.ui;

public class AndroidForm extends MobileForm {
    @Override
    protected MobileElement createElement(String title) {
        return new AndroidElement(title);
    }
}

// mobileapplication/ui/IOSForm.java
package mobileapplication.ui;

public class IOSForm extends MobileForm {
    @Override
    protected MobileElement createElement(String title) {
        return new IOSElement(title);
    }
}

// mobileapplication/App.java
package mobileapplication;

import mobileapplication.ui.AndroidForm;
import mobileapplication.ui.IOSForm;
import mobileapplication.ui.MobileForm;

public class App {
    private MobileForm form;

    private String getOS() {
        return Math.random() > 0.5 ? "Android" : "iOS"; // Platzhalter
    }

    private void initialize() {
        String os = getOS();
        form = "Android".equals(os) ? new AndroidForm() : new IOSForm();
    }

    public void execute() {
        initialize();
        form.addElement("Header");
        form.addElement("Content");
        form.addElement("Footer");
        form.render();
    }
}
\end{minted}
Kernideen: Factory Method (\texttt{createElement}) koppelt \texttt{App} von den konkreten OS-Klassen ab und stellt sicher, dass Form und Elemente stets zusammenpassen. Kapselung über paket-private Konstruktoren verhindert die Erzeugung konkreter Elemente außerhalb von \texttt{mobileapplication.ui}.

\end{document}
