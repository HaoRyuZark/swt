\documentclass[12pt]{article}

% -----------------------------
% Sprache, Encoding & Fonts
% -----------------------------
\usepackage[utf8]{inputenc}
\usepackage[T1]{fontenc}
\usepackage[ngerman]{babel}
\usepackage{csquotes}     % \enquote
\usepackage{microtype}    % bessere Laufweite/Umbruch
\usepackage{ebgaramond}

% -----------------------------
% Layout & Grundpakete
% -----------------------------
\usepackage{geometry}
\geometry{margin=1in}
\usepackage{setspace}
\usepackage{parskip}

% -----------------------------
% Mathematik & Symbole
% -----------------------------
\usepackage{amsmath,amssymb,amsthm}
\usepackage{mathrsfs}

% -----------------------------
% Grafiken & Plots
% -----------------------------
\usepackage{graphicx}
\graphicspath{{./resources/images/}}
\usepackage{caption}
\usepackage{subcaption}
\usepackage{float}
\usepackage{pgfplots}
\pgfplotsset{compat=1.18}
\usepackage{tikz}
\usetikzlibrary{calc,intersections,decorations.pathreplacing,patterns,angles,quotes,shapes,arrows,positioning,automata}
\usepgfplotslibrary{fillbetween}
\usepackage[output=png]{plantuml} % <— PlantUML für UML-Diagramme

% -----------------------------
% Sonstige Nützliches
% -----------------------------
\usepackage{enumitem}
\usepackage{booktabs}
\usepackage{multicol}
\usepackage{multirow}
\usepackage{polynom}
\usepackage{tikz-cd}
\usepackage{comment}
\usepackage{fancyvrb}
\usepackage{fancyhdr}
\usepackage{etoolbox}
\usepackage{listings}
\usepackage{tkz-euclide}
\usepackage{hyperref}
\usepackage{siunitx}   % für circuitikz/Si-Einheiten
\usepackage{circuitikz}
\usepackage{bookmark}
\usepackage{tabularx}
\usepackage{ragged2e}  % für RaggedRight in Tabellenspalten

% -----------------------------
% Tabellen: neue Spaltenart (ragged) für bessere Umbrüche
% -----------------------------
\newcolumntype{Y}{>{\RaggedRight\arraybackslash}X}

% -----------------------------
% Header/Footer
% -----------------------------
\pagestyle{fancy}
\fancyhead[l]{Gruppe 12}
\fancyhead[c]{SWT \#5}
\fancyhead[r]{\today}
\fancyfoot[c]{\thepage}
\renewcommand{\headrulewidth}{0.2pt}
\setlength{\headheight}{15pt}

% Etwas mehr Flexibilität beim Zeilenumbruch
\emergencystretch=1.5em

\begin{document}

\section*{A1 Continuous Deployment}

Common Risks von CD: 

\begin{itemize}

\item Falls das Test-Environment nicht schnell angepasst werden kann, kann es beispielsweise zur wiederholten Auslieferung fehlerhafter Software kommen, wodurch sich Bugs akkumulieren. Dies kann zu einem erheblichen Problem werden, insbesondere wenn diese Bugs aus aufeinander aufbauenden Features entstehen.

\item Sollte Malware-Code infolge eines Angriffs in das Projekt gelangen, sind die Kunden schneller betroffen als bei Continuous Delivery, da die Auslieferungen häufiger stattfinden.

\item Ein Fehler in der Pipeline – insbesondere im Bereich der Auslieferung von Artefakten – kann durch fehlerhafte Konfigurationen ebenfalls zu einer schnellen Verbreitung von Fehlern in den Kundensystemen führen.

\end{itemize}

\section*{A3 Online-Reservierungssystem für Impfungen}
\begin{plantuml}
@startuml
title Online-Reservierungssystem für Impfungen – Nachrichtenfluss (Patient ↔ Web-Anwendung ↔ Dienste)

' ——— ISO/UML-nahe Darstellungseinstellungen ———
hide footbox
autonumber
skinparam shadowing false
skinparam monochrome true
skinparam defaultFontName Arial
skinparam ArrowThickness 1
skinparam ParticipantBorderColor Black
skinparam ParticipantBackgroundColor White
skinparam ActorBorderColor Black
skinparam ActorBackgroundColor White
skinparam SequenceLifeLineBorderColor Black
skinparam SequenceLifeLineBackgroundColor White
skinparam NoteBackgroundColor White
skinparam sequence {
  MessageAlign left
  ParticipantPadding 10
  BoxPadding 10
}

actor "Patient" as P
participant "Web-Anwendung\n<<boundary/GUI>>" as GUI
participant "Identitätsmanagement\n<<service>>" as IDP
participant "Termin-Verwaltungssystem\n<<service>>" as SCHED

' ——— Login ———
P -> GUI : Seite aufrufen &\nAnmeldedaten eingeben (Benutzername, Passwort)
activate GUI
GUI -> IDP : validateCredentials(benutzername, passwort)
activate IDP

alt Ungültige Anmeldedaten
  IDP --> GUI : authFailed
  deactivate IDP
  GUI -> P : Passwort-Reset anbieten
  GUI -> IDP : initiatePasswordReset(benutzername)
  deactivate GUI
else Gültige Anmeldedaten
  IDP --> GUI : authOK(nutzerId)
  deactivate IDP

  ' ——— Tage abfragen ———
  GUI -> SCHED : getAvailableDays()
  activate SCHED
  SCHED --> GUI : tage[]
  deactivate SCHED
  GUI -> P : Mögliche Tage anzeigen
  P -> GUI : Tag auswählen (tag)

  ' ——— Uhrzeiten abfragen ———
  GUI -> SCHED : getAvailableTimes(tag)
  activate SCHED
  SCHED --> GUI : uhrzeiten[]
  deactivate SCHED
  GUI -> P : Mögliche Uhrzeiten anzeigen
  P -> GUI : Uhrzeit wählen & Buchung bestätigen (zeit)

  ' ——— Termin reservieren + Kalender ———
  GUI -> SCHED : reserveAppointment(tag, zeit, nutzerId)
  activate SCHED
  SCHED --> GUI : reservationConfirmed(terminId, details)
  deactivate SCHED

  GUI -> IDP : addAppointmentToCalendar(terminId, nutzerId, details)
  activate IDP
  IDP --> GUI : calendarUpdated

  opt E-Mail-Benachrichtigung angefordert
    GUI -> IDP : sendEmailConfirmation(nutzerEmail, details)
    IDP --> P : E-Mail mit Termindaten
  end
  deactivate IDP

  GUI -> P : Buchungsbestätigung anzeigen
  deactivate GUI
end

@enduml
  
\end{plantuml}


\section*{A2 Klassich oder Agil}

Für ein Projekt im medizinischen Bereich oder in anderen kritischen Bereichen sollte Scrum nicht verwendet werden. Stattdessen sollte klassisch 
vorgegangen werden.Die Gründe für diese Entscheidung sind, dass sowohl die Daten als auch die Tätigkeiten in diesen Gebieten 
hochsensibel sind. Außerdem deutet die Beschreibung des Dienstes auf die Digitalisierung eines bereits vorhandenen, funktionierenden Systems mit fest geplanten Funktionen hin. 
In einem solchen Fall kann die Agilität von Scrum nicht wirklich hilfreich sein, da die Features des Produkts von Anfang an klar definiert sind.

In der Medizin spielt auch die juristische Schuldfrage eine wesentliche Rolle. Da in Scrum die Verantwortlichkeit nicht besonders stark berücksichtigt wird, kann dies im 
Fall eines kritischen Fehlers im System zu juristischen Problemen für Arztpraxen und Krankenhäuser führen – auch dann, wenn das Problem tatsächlich auf Seiten der Entwickler oder des 
Unternehmens entstanden ist.

\section*{A4 Auto starten}


\end{document}
