\documentclass[12pt]{article}

% -----------------------------
% Sprache, Encoding & Fonts
% -----------------------------
\usepackage[utf8]{inputenc}
\usepackage[T1]{fontenc}
\usepackage[ngerman]{babel}
\usepackage{csquotes}
\usepackage{microtype}
\usepackage{ebgaramond}

% -----------------------------
% Layout & Grundpakete
% -----------------------------
\usepackage{geometry}
\geometry{margin=1in}
\usepackage{setspace}
\usepackage{parskip}

% -----------------------------
% Mathematik & Symbole
% -----------------------------
\usepackage{amsmath,amssymb,amsthm}
\usepackage{mathrsfs}

% -----------------------------
% Sonstige Nützliches
% -----------------------------
\usepackage{graphicx}
\usepackage{enumitem}
\usepackage{booktabs}
\usepackage{tabularx}
\usepackage{ragged2e}
\usepackage{fancyhdr}
\usepackage{hyperref}
\usepackage{minted}

% -----------------------------
% Tabellen: neue Spaltenart
% -----------------------------
\newcolumntype{Y}{>{\RaggedRight\arraybackslash}X}

% -----------------------------
% Header/Footer
% -----------------------------
\pagestyle{fancy}
\fancyhead[l]{Gruppe 12}
\fancyhead[c]{SWT \#8}
\fancyhead[r]{\today}
\fancyfoot[c]{\thepage}
\renewcommand{\headrulewidth}{0.2pt}
\setlength{\headheight}{15pt}

\emergencystretch=1.5em

\begin{document}

\section*{Aufgabe A1 -- Mobile Application}

\subsection*{a) Klassenverband (Factory Method)}
\paragraph{Klassen im Paket \texttt{mobileapplication.ui}}
\begin{description}[leftmargin=*,style=nextline]
    \item[\textbf{MobileElement} (abstrakt, Product)]
    \begin{itemize}[leftmargin=1.5em,noitemsep,topsep=0pt]
        \item schreibgeschütztes Attribut \texttt{title : String}
        \item Referenz auf die zugehörige \texttt{MobileForm}
        \item Operationen: \texttt{render()} (abstrakt), \texttt{register(MobileForm form)}, \texttt{getTitle()}
    \end{itemize}

    \item[\textbf{MobileForm} (abstrakt, Creator)]
    \begin{itemize}[leftmargin=1.5em,noitemsep,topsep=0pt]
        \item Aggregation von \texttt{List<MobileElement> elements} (Teile-Ganze-Beziehung)
        \item Factory Method \texttt{createElement(String title)} (\textit{protected abstract})
        \item konkrete Methode \texttt{addElement(String title)}: ruft \texttt{createElement}, registriert das Element bei der Form und fügt es \texttt{elements} hinzu
        \item \texttt{render()} delegiert an \texttt{render()} aller enthaltenen Elemente
    \end{itemize}

    \item[\textbf{AndroidForm} / \textbf{IOSForm} (konkrete Creator)]
    \begin{itemize}[leftmargin=1.5em,noitemsep,topsep=0pt]
        \item erben von \texttt{MobileForm}
        \item implementieren \texttt{createElement(...)} und erzeugen jeweils passende konkrete Elemente
    \end{itemize}

    \item[\textbf{AndroidElement} / \textbf{IOSElement} (konkrete Products)]
    \begin{itemize}[leftmargin=1.5em,noitemsep,topsep=0pt]
        \item erben von \texttt{MobileElement}
        \item Konstruktoren mit Paket-Sicht, sodass Instanzen nur innerhalb von \texttt{mobileapplication.ui} erzeugt werden können
        \item implementieren \texttt{render()} betriebssystemspezifisch
    \end{itemize}
\end{description}

\paragraph{Klasse im Paket \texttt{mobileapplication}}
\begin{description}[leftmargin=*,style=nextline]
    \item[\textbf{App}]
    \begin{itemize}[leftmargin=1.5em,noitemsep,topsep=0pt]
        \item Attribut \texttt{form : MobileForm}
        \item private Operation \texttt{getOS()} liefert \enquote{Android} oder \enquote{iOS}
        \item private Operation \texttt{initialize()} erzeugt passende konkrete Form (\texttt{AndroidForm} oder \texttt{IOSForm}) und setzt \texttt{form}
        \item \texttt{execute()} ruft \texttt{initialize()}, fügt drei neue Elemente über \texttt{form.addElement(...)} hinzu und ruft anschließend \texttt{form.render()}
    \end{itemize}
\end{description}

\paragraph{UML-Beziehungen (textuelle Skizze)}
\begin{itemize}[leftmargin=*]
    \item \texttt{MobileElement} \texttt{$\lhd$--} \texttt{AndroidElement}, \texttt{IOSElement} (Vererbung; abstraktes Product).
    \item \texttt{MobileForm} \texttt{$\lhd$--} \texttt{AndroidForm}, \texttt{IOSForm} (Vererbung; abstrakter Creator mit Factory Method \texttt{createElement}).
    \item \texttt{MobileForm} \texttt{$\blacklozenge$--} \texttt{MobileElement} (Komposition über \texttt{elements}).
    \item \texttt{App} \texttt{--} \texttt{MobileForm} (Assoziation über Attribut \texttt{form}); \texttt{App} nutzt nur die abstrakten Typen und ist nicht an konkrete OS-Klassen gekoppelt.
\end{itemize}

\begin{figure}[htbp]
    \centering
    \includegraphics[width=0.9\linewidth]{SWT8-class-plantuml.png}
    \caption{UML-Klassendiagramm für das Mobile-Application-Framework (modelliert mit PlantUML)}
\end{figure}

\subsection*{b) Sequenzdiagramm (Android)}
Gefundene Nachricht: \texttt{App.execute()}.

\begin{figure}[htbp]
    \centering
    \includegraphics[width=0.9\linewidth]{SWT8-seq-plantuml.png}
    \caption{UML-Sequenzdiagramm der Ausführung von \texttt{App.execute()} auf Android (modelliert mit PlantUML)}
\end{figure}

\subsection*{c) Java-Rumpf (Paketstruktur angedeutet)}
\begin{minted}[fontsize=\footnotesize,breaklines,linenos]{java}
// mobileapplication/ui/MobileElement.java
package mobileapplication.ui;

public abstract class MobileElement {
    private final String title;
    private MobileForm form;

    protected MobileElement(String title) { // nur Paket/SUB-Klassen
        this.title = title;
    }

    public String getTitle() { return title; }

    void register(MobileForm form) { // Paket-sichtbar
        this.form = form;
    }

    public MobileForm getForm() { return form; }

    public abstract void render();
}

// mobileapplication/ui/MobileForm.java
package mobileapplication.ui;

import java.util.ArrayList;
import java.util.List;

public abstract class MobileForm {
    private final List<MobileElement> elements = new ArrayList<>();

    protected abstract MobileElement createElement(String title);

    public final void addElement(String title) {
        MobileElement element = createElement(title);
        element.register(this);
        elements.add(element);
    }

    public void render() {
        elements.forEach(MobileElement::render);
    }
}

// mobileapplication/ui/AndroidElement.java
package mobileapplication.ui;

class AndroidElement extends MobileElement {
    AndroidElement(String title) { super(title); }

    @Override
    public void render() {
        System.out.println("Android-Element: " + getTitle());
    }
}

// mobileapplication/ui/IOSElement.java
package mobileapplication.ui;

class IOSElement extends MobileElement {
    IOSElement(String title) { super(title); }

    @Override
    public void render() {
        System.out.println("iOS-Element: " + getTitle());
    }
}

// mobileapplication/ui/AndroidForm.java
package mobileapplication.ui;

public class AndroidForm extends MobileForm {
    @Override
    protected MobileElement createElement(String title) {
        return new AndroidElement(title);
    }
}

// mobileapplication/ui/IOSForm.java
package mobileapplication.ui;

public class IOSForm extends MobileForm {
    @Override
    protected MobileElement createElement(String title) {
        return new IOSElement(title);
    }
}

// mobileapplication/App.java
package mobileapplication;

import mobileapplication.ui.AndroidForm;
import mobileapplication.ui.IOSForm;
import mobileapplication.ui.MobileForm;

public class App {
    private MobileForm form;

    private String getOS() {
        return Math.random() > 0.5 ? "Android" : "iOS"; // Platzhalter
    }

    private void initialize() {
        String os = getOS();
        form = "Android".equals(os) ? new AndroidForm() : new IOSForm();
    }

    public void execute() {
        initialize();
        form.addElement("Header");
        form.addElement("Content");
        form.addElement("Footer");
        form.render();
    }
}
\end{minted}
Kernideen: Factory Method (\texttt{createElement}) koppelt \texttt{App} von den konkreten OS-Klassen ab und stellt sicher, dass Form und Elemente stets zusammenpassen. Kapselung über paket-private Konstruktoren verhindert die Erzeugung konkreter Elemente außerhalb von \texttt{mobileapplication.ui}.

\end{document}
